\section{Toy Problem Description} \label{sec:problem-description}

A simple problem is used to investigate the genotype-phenotype map generation techniques proposed.
This problem, the $n$-legged table problem, models a table design scenario
in which stable tables are highly advantageous.
In this scenario, the phenotype of a table is nothing more than a collection of continuous-valued individual leg lengths.
All other details of table design are neglected.
The stability of a table is assumed to result solely from uniform lengths between table legs.
Clearly, as $n$ grows beyond eight or so this toy problem begins to lose a meaningful connection to real world tables.
(When was the last time you saw a fifty-legged table?)
However, mathematically (and intuitively) the $n$-legged table problem scales easily.
We arbitrarily use $n=100$ for all experiments.

This toy problem was chosen because it creates an easy-to-characterize rugged fitness landscape.
Because unstable tables are disadvantageous, mutations to level tables tend to be deleterious.
Thus, evolving between different table heights --- i.e. escaping local maxima --- becomes a tricky challenge.

The details of specific evaluation criteria, phenotype representations, and genotype representations used for this toy problem can be found in Section \ref{sec:methods}.
