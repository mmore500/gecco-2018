\section{Toy Problem Description} \label{sec:problem-description}

A simple problem will be used as proof of concept.
This problem, the $n$-legged table problem, models a table design scenario where selection is made for the height of the table and it is highly disadvantageous to have different-length legs.
In this case, short tables will be selected for over tall tables.

In the phenotype space, solutions are a $n$-length bitstring.
The $i$th bit of the bitstring represents whether the the $i$th leg of the table is in the long or short configuration.

In this problem, phenotypic novelty can be measured by the hamming distance between two phenotypes.

The cost function will be defined as,
\begin{align*}
C(l_1, l_2, \ldots, l_n)
&=
\alpha \sum_{i = 1}^{n} l_i
+
\sum_{i=1}^n \Big[l_i - \Big(\frac{1}{n} \sum_{j=1}^n l_j\Big)\Big]^2
\end{align*}
where $\alpha$ is a parameter that controls how strongly favored short tables are over tall tables.
Fitness will be inversely proportional to cost.
As selection will be performed by tournament, only relative --- not absolute fitness --- is of concern.

Suppose $n=4$ and $\alpha = \frac{1}{10}$.
Then, we can tabulate the cost for having different leg combinations
\begin{center}
\begin{tabular}{ c c c }
 $C$ & short legs & long legs \\
 0.20 & 0 & 4 \\
 0.40 & 1 & 3 \\
 0.43 & 2 & 2 \\
 0.30 & 3 & 1 \\
 0.00 & 4 & 0
\end{tabular}
\end{center}
Thus, we can see that the fitness landscape is deceptive.
Although the global minimum cost occurs with four short legs, initially converting long legs to short legs results in an increased cost.
A similar calculation shows that increasing alpha decreases the deceptiveness of the search space.
