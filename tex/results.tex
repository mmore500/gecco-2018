\section{Results and Discussion} \label{sec:results}

Assessment of evolvability signatures and observed response to selective pressure towards a global fitness peak shows that both autoencoder-derived genotype-phenotype maps enhance evolvability relative to the direct encoding.

TODO


\subsection{$n$-legged Table Problem}

\begin{figure}
        \begin{subfigure}[b]{0.33\linewidth}
                \includegraphics[width=\linewidth]{img/direct_es_unscaled}
        \end{subfigure}%
        \begin{subfigure}[b]{0.33\linewidth}
                \includegraphics[width=\linewidth]{img/bottleneck_es_unscaled}
        \end{subfigure}%
        \begin{subfigure}[b]{0.33\linewidth}
                \includegraphics[width=\linewidth]{img/noise_es_unscaled}
        \end{subfigure}
        \caption{Evolvability signatures for, left to right, direct, bottleneck, and denoising genotype-phenotype maps.}\label{fig:all_es}
\end{figure}


Evolvability signature analysis was used to characterize the evolvability of the learned $n$-legged table bottleneck and denoiser encodings.
To generate these signatures, 40 replicate populations were evolved for 50 generations.
Then, the individual with the best fitness score over those 50 generations was isolated.
The novelty (euclidean distance between a parent phenotype and a child phenotype) and fitness scores of mutant offspring generated from that champion individual were used to create the evolvability signatures.
Figure \ref{fig:all_es} provides evolvability signatures for the direct, bottleneck, and denoiser genotype-phenotype maps.
It must be noted that in Figure \ref{fig:all_es}, the evolvability signature for each genotype-phenotype map are at radically different absolute scales.
It can clearly be seen that the bottleneck mapping can generate much more novelty per mutational step than either of the other mappings.
Note that the absolute fitness scores of nearly all offspring under the bottleneck mapping are greater than the absolute fitness of offspring under the direct mapping.
Although the novelty generated under the denoiser encoding via a single mutational step is comparable to the direct encoding, the absolute fitness of offspring under this mapping tends to be greater than those under the direct mapping.

Next, experiments were performed to assess the ability of the three genotype-phenotype maps to facilitate traversal of the evolutionary search space in response to a selective pressure.
Specifically, a selective pressure for short (i.e. zero) table height was added.
Table height is calculated as the mean leg length of a table.
For each map, three replicate populations of 300 individuals were evolved for 5,000 generations.
These populations were initialized to have table height of approximately 1,000.
Response to selective pressure for short table height was assessed by tracking mean table height of the populations generation-by-generation.

\begin{figure}
  \includegraphics[width=\linewidth]{img/zero_leg_selection}
  \caption{Response to short-table selection pressure under different genotype-phenotype maps.}
  \label{fig:select_response}
\end{figure}


Figure \ref{fig:select_response} plots mean table height by generation under selection for both zero table height and table stability (described in detail in Section \ref{sec:methods}).
Shaded areas, hugging the table height trajectories so tight as to be difficult to discern, represent bootstrapped 95\% confidence intervals ($n=3$).

Under the direct mapping, a slight decrease in mean table height is observed for a few generations after initialization.
However, no further decrease in mean table height was observed over the course of evolutionary runs.
These runs ended with a mean table height of approximately 950.
Direct-encoded populations were trapped at local fitness peaks and unable to respond to selective pressure for short table height.

Under the bottleneck mapping, a severe decrease in mean table height is observed after initialization.
Well within 100 generations, the populations have come close to the global fitness peak --- a level table with height 0.
Populations with the bottleneck mapping were able to quickly respond to selective pressure for short table height.

Finally, under the denoising mapping, a slight drop-off in mean table height is observed after initialization.
Then, a steady decrease in table height was observed for the remainder of the evolutionary runs.
These runs ended with a mean table height of approximately 630.
Although not as swiftly as under the bottleneck mapping, under the denoising mapping populations were still able to respond to selective pressure for short table height.

Cursory inspection of input/output of the decoding component of the bottleneck autoencoder revealed that it had learned to echo the single bottlenecked float in order to uniformly populate the vector of 100 phenotypic leg lengths.
Thus, the bottleneck genetic representation essentially described the table height; table leg lengths were set using this height value via the genotype-phenotype map.
Evolvability signature analysis suggests that this encoding allows the generation of greater novelty with less deleterious consequence.
Indeed, as in Figure \ref{fig:select_response}, table populations using this encoding are able to rapidly evolve to a global fitness peak.
Similar inspection of the denoising autoencoder revealed that that neural network had essentially learned to calculate each phenotypic leg length output as the average of its inputs.
Thus, the denoising encoding took genetic input for what --- under the direct encoding --- would otherwise be an unstable table and output a stable table nearby in phenotype space.
Evolvability signature analysis shows that although this encoding does not enhance the novelty of phenotypic outcomes under mutation relative to the direct encoding, it does curb the deleterious effects of mutation.
Thus, as in Figure \ref{fig:select_response}, table populations using this encoding are able to make slow and steady progress towards a global fitness peak.

\subsection{Scrabble String Problem}
