\section{Results} \label{sec:results}

Assessment of evolvability signatures and observed response to selective pressure towards a global fitness peak shows that both autoencoder-derived genotype-phenotype maps enhance evolvability relative to the direct encoding.

\subsection{Evolvability Signatures}

\begin{figure}
        \begin{subfigure}[b]{0.33\linewidth}
                \includegraphics[width=\linewidth]{img/direct_es_unscaled}
        \end{subfigure}%
        \begin{subfigure}[b]{0.33\linewidth}
                \includegraphics[width=\linewidth]{img/bottleneck_es_unscaled}
        \end{subfigure}%
        \begin{subfigure}[b]{0.33\linewidth}
                \includegraphics[width=\linewidth]{img/noise_es_unscaled}
        \end{subfigure}
        \caption{Evolvability signatures for, left to right, direct, bottleneck, and denoising genotype-phenotype maps.}\label{fig:all_es}
\end{figure}

\begin{figure}
        \begin{subfigure}[b]{0.33\textwidth}
                \includegraphics[width=\linewidth]{img/direct_es_scaled}
        \end{subfigure}%
        \begin{subfigure}[b]{0.33\textwidth}
                \includegraphics[width=\linewidth]{img/bottleneck_es_scaled}
        \end{subfigure}%
        \begin{subfigure}[b]{0.33\textwidth}
                \includegraphics[width=\linewidth]{img/noise_es_scaled}
        \end{subfigure}
        \caption{Same-scale evolvability signatures for, left to right, direct bottleneck, and denoising genotype-phenotype maps.}
        \label{fig:all_es_scaled}
\end{figure}

\begin{figure}
        \begin{subfigure}[b]{0.5\textwidth}
                \includegraphics[width=\linewidth]{img/direct_severe_es_scaled2}
        \end{subfigure}%
        \begin{subfigure}[b]{0.5\textwidth}
                \includegraphics[width=\linewidth]{img/noise_severe_es_scaled2}
        \end{subfigure}
        \caption{
          Same-scale severe evolvability signatures for, left to right, direct and denoising genotype-phenotype maps.
          }\label{fig:noise_severe_compare_es}
\end{figure}



Figure \ref{fig:all_es} provides evolvability signatures for the direct, bottleneck, and denoiser genotype-phenotype maps.
These evolvability signatures are heat maps that summarize the outcomes observed under mutation for

As is expected, in all the direct and bottleneck evolvability signatures, offspring fitness tends to decrease somewhat as is expected.
Surprisingly, under the denoising mapping this is not true.
This suggests that the denoising mapping is better able to generate novelty without losing of fitness or, put another way, that mutation tends to be mildly deleterious.

It must be noted that in Figure \ref{fig:all_es}, the evolvability signature for each genotype-phenotype map are at radically different absolute scales.
Figure \ref{fig:all_es_scaled.tex} compares the evolvability signatures of the three genotype-phenotype maps at the same absolute scale.
It can clearly be seen that the bottleneck mapping can generate much more novelty per mutational step than either of the other mappings.
Note that the absolute fitness scores of nearly all offspring under the bottleneck mapping are greater than the absolute fitness of offspring under the direct mapping.
The same is true of the denoising mapping.

Because mutation under the direct and denoising mappings tend to have relatively small phenotypic effects, ``severe'' evolvability signatures were generated to better compare the evolvability of these two mappings.
These severe signatures, shown in Figure \ref{fig:noise_severe_compare_es}, are generated as before, except instead of applying a single mutation to generate mutant offspring from a parent 100 mutations were applied.
Thus, these charts reflect the novelty and fitness outcomes of larger mutational steps in the genotype space.
From this comparison, it can be seen that somewhat greater novelty tends to be generated under the direct mapping.
However, much better fitness outcomes are observed with the denoising mapping.
Again, novelty seems to be generated with little or no fitness cost.
Thus, the denoising mapping appears to produce more useful variation than the direct mapping.

\subsection{Response to Selection}

\begin{figure}
  \includegraphics[width=\linewidth]{img/zero_leg_selection}
  \caption{Response to short-table selection pressure under different genotype-phenotype maps.}
  \label{fig:select_response}
\end{figure}


Figure \ref{fig:select_response} plots mean table height by generation under selection for both short table height and table stability (described in detail in Section \ref{sec:methods}).
Error bars representing standard deviation between three replicate runs, although so minuscule as not to be easily visible, are provided every 1,000 generations.

Under the direct mapping, a slight decrease in mean table height is observed for a few generations after initialization.
However, no further decrease in mean table height was observed over the course of evolutionary runs.
These runs ended with a mean table height of approximately 950.
Direct-encoded populations were trapped at local fitness peaks and unable to respond to selective pressure for short table height.

Under the bottleneck mapping, a severe decrease in mean table height is observed after initialization.
Well within 100 generations, the populations have come close to the global fitness peak --- a level table with height 0.
Populations with the bottleneck mapping were able to quickly respond to selective pressure for short table height.

Finally, under the denoising mapping, a slight drop-off in mean table height is observed after initialization.
Then, a steady decrease in table height was observed for the remainder of the evolutionary runs.
These runs ended with a mean table height of approximately 630.
Although not as quickly as under the bottleneck mapping, under the denoising mapping populations were still able to respond to selective pressure for short table height.
