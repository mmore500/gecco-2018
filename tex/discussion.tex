\section{Discussion} \label{sec:discussion}

Evolvability signature analysis indicate that, in the $n$-leg table domain, the bottleneck and denoising mapping are both more evolvable than the direct encoding.
The bottleneck mapping allows large mutational steps to be taken through the phenotype space relative to the denoising mapping and the direct mapping.
Both tend to have mutational outcomes that are superior in fitness to those of the direct encoding.
The bottleneck and denoising mappings were observed to free evolution from getting stuck at local fitness peaks, as occurred under the direct mapping, and instead enable progress towards a global fitness peak.
Thus, in the $n$-leg table domain, the proposed genotype-phenotype maps seem to enhance evolvability in both a practical and theoretical sense.

Hand-design of genotype-phenotype maps is common practice in evolutionary computing.
These genotype-phenotype mappings, such as NEAT, have been used with good success and studied extensively \cite{stanley2002efficient}.
In particular, there has been interest in designing dynamic genotype-phenotype mappings that are influenced by the contents of a genome, as occurs in nature, so that an evolvable mapping may themselves be evolved \cite{reisinger2007acquiring}.
Nonetheless, existing genotype-phenotype mappings can be domain-specific;
a scheme useful for artificial neuroevolution, for example, likely isn't useful in a linear genetic programming context.

The proposed techniques to generate evolvable genotype-phenotype mappings are in principal generalizable to any context in which many phenotypes at local fitness peaks can be surveyed (i.e. extensive exploratory evolution with a direct encoding isn't computationally prohibitive) and phenotypes can be represented as a continuous-valued, constant-length vector.
Using these techniques may reduce the human labor and expertise required to design evolvable genotype-phenotype mappings for new evolutionary computing domains.
Further, these autoencoder-based approaches might yield more evolvable genotype-phenotype maps than human design for existing evolutionary computing domains.
However, it must be noted that the proposed techniques have only been demonstrated on a toy problem for which it is trivial to manually design an evolvable genotype-phenotype mapping.
Further, it is certainly true that for more complex applications, autoencoder design itself typically requires skilled human input.
Thus, these automatic genotype-phenotype mappings cannot entirely sidestep the need for manual labor.
In fact, in more complex domains, these techniques will introduce a new cost: computation.
Training autoencoders in complex domains, especially the process of developing autoencoder design, will require significant computational resources.

Future work with these autoencoder-based genotype-phenotype maps will focus on demonstrating their utility in more complex domains.
It will be interesting to research how to adapt autoencoder architecture and training to a domain less well-understood than the $n$-leg problem.
Ultimately, work should apply these techniques to a domain where human-designed genotype-phenotype mappings have significant shortcomings to attempt to surpass the performance of human-designed maps.
