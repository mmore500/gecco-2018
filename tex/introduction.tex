\section{Introduction} \ref{sec:introduction}

Successful evolutionary search depends the production of heritable, novel phenotypic variation, some of which must not be severely deleterious.
Without any heritable variation --- or even just without any viable heritable variation --- evolution stagnates.
The capacity of a population to generate viable heritable phenotypic variation is referred to as evolvability.
Different evolving systems can exhibit different degrees of evolvability.
Natural systems, in particular, are thought to generally exhibit greater evolvability than digital evolution systems \cite{wagner1996perspective}.
Understanding --- and replicating --- the evolvability of natural evolution is an open problem in digital evolution research \cite{mengistu2016evolvability}.

Evolvability is desirable in artificial evolution systems for practical ends --- developing more evolvable artificial evolution systems will allow evolutionary algorithms to tackle sophisticated problems more effectively and efficiently \cite{bentley1999three, reisinger2007acquiring}.
In addition, understanding evolvability is additionally of great scientific interest for evolutionary biologists and evolutionary computing researchers alike \cite{mengistu2016evolvability, pigliucci2008evolvability}, particularly with respect to the addressing questions related to the evolution of complexity and open-ended evolution \cite{kirschner1998evolvability, hu2010evolvability}.

Indeed, there has been great interest in studying evolvability using computational systems and, in particular, developing techniques to promote evolvability in digital evolution \cite{kashtan2005spontaneous, mengistu2016evolvability, reisinger2005towards, cheney2013unshackling, nguyen2015innovation, lehman2013evolvability}.
Inspired by recent theoretical advances applying learning theory to the topic of evolvability \cite{kouvaris2017evolution, watson2016can}, we propose methodology based on autoencoder artificial neural networks that allows evolvable genotype-phenotype encodings to be learned by training on phenotypes harvested from local fitness peaks.
We call our approach AutoMap.
One variant of AutoMap employs a denoising autoencoder to learn a representation that buffers phenotypes near local fitness peaks against mutation until a mutational threshold is reached where the phenotype shifts to the vicinity of a different local fitness peak.
The second AutoMap variant employs a bottlenecked autoencoder to learn a representation where small steps in the genotype space yield significant phenotypic novelty while protecting phenotypic viability.
In principle, the methodology introduced is general enough to apply across a wide variety of digital evolution problem domains.
