\section{Approach} \label{sec:approach}

The algorithm will consist of alternating periods of evolving in the solution space and evolving in the representation space.
For both representations, tournament selection will be used and recombination will consist solely of mutation.

As described in Section \ref{sec:problem}, the solution genotype and phenotype will consist of a bitstring.
The mapping genotype will enumerate a mapping between solution genotype and phenotype via a lookup table.
All solution genotypes, all of which must be valid, will form the left-hand side of the lookup table.
These entries will not be considered part of the mapping genotype and therefore will not be subject to recombination.
The right-hand side of the lookup table will serve as the genotype of the mapping.
The genotype of the mapping will be represented as a bitstring.
For a solution genotype of length $n$, the mapping genotype will be of length
$2^n \times n$.

The phenotype of the mapping will be expressed by
\begin{enumerate}
\item converting the solution genotype bitstring into an integer value $i$,
\item indexing into the $i$th set of $n$ bits in the mapping genotype, and
\item taking that set of bits as the solution phenotype.
\end{enumerate}

Fitness of a mapping will be calculated by generating a set of mutant offspring from the currently existing population of solution genotypes and calculating the
\begin{enumerate}
\item novelty of mutant offspring with respect to their parent, and
\item fitness of mutant offspring with respect to their parent
\end{enumerate}
Specifically, mapping fitness will calculated as a weighted sum of these two criteria.
Solution novelty metrics and fitness criteria are described in Section \ref{sec:problem}.
