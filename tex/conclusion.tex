We introduced AutoMap, a scheme to learn evolvable genotype-phenotype encodings via autoencoder artificial neural networks trained on champion phenotypes from a preliminary evolutionary search.
We presented two variants of the AutoMap approach.
One is built around a denoising autoencoder that yields a more neutral evolutionary search space through phenotypic robustness under mutation.
The other employs a bottleneck autoencoder to yield a more compact representation
Both approaches were demonstrated to increase evolvability relative to the direct encoding in the context of a proof-of-concept test problem, the $n$-legged table problem.
The denoiser approach was further demonstrated to increase evolvability in the more challenging Scrabble string problem.

Much work is left to be done.
As immediate next steps, the bottleneck approach should be demonstrated in the Scrabble string domain and the capacity of both approaches to facilitate evolution towards higher-value fitness peaks in the Scrabble string domain --- perhaps by defining high-level criteria to be rewarded in addition to the low-level spelling criteria --- should be explored.
In addition, this approach should be put in explicit conversation with recent thinking about parallels between the principles of machine learning and the evolution of evolvability \cite{kouvaris2017evolution, watson2016can}.

This exploratory work only scratches the surface of the possible applications of the AutoMap approach.
In principle, artificial neural network autoencoders can be employed with any phenotype that can be represented as a vector.
However, it must be recognized that the success of the approach in any problem domain is predicated on design of an appropriate encoder and training scheme, the computational capacity to generate large amounts of training data through direct evolution, and acceptance of the computational cost of performing a forward pass through the autoencoder component for each fitness evaluation when evolving with a learned genotype-phenotype map.
Despite these limitations, it is hoped that the AutoMap approach to learning evolvable genotype-phenotype maps will prove a useful component of the digital evolution toolbox.
